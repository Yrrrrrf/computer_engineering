% DON'T USE TAB. TAB BREAKS MUST OF THE \LaTeX CODE.
\documentclass{article}

\usepackage{amsmath}

\title{Basic amsmath package Manipulation}
\author{Yrrrrrf}
\date{July 7th, 2020}

\begin{document}

    \maketitle

    \normalsize Hi Math Symbols! \\ \raggedright
    There are two math modes int \LaTeX. \\
    Display Style Math that'll put the equations below the text. \\
    Inline/Text Style Math, that'll put math equations in the same line as the text.
    \\[\baselineskip]
    All equations are \textbf{centered} by default.
    \\[2\baselineskip]

    \section{Display Style Math}
        % Is centerred by default.
        Puts math on below the text.

        % \[equation \]
        \[f(x) = (x+2)^3 - 9/3 + \frac{x}{2} \] \\

        % Display math environment.
        Used when we want to many equations consecutively.
        \begin{align*}
            % AND (&) operator tell LaTeX where the alignment will be.
            f(x,y) & = a^2 + 2ab + b^2 \\
                & = (a+b)^2 \\
        \end{align*}

        Using the star will make the equations automaticly numbered.
        \begin{align}
            % This can also be used many times in a same line.
            2x+1 & = 9 & 3y+x & =13\\
            2x & =8 & 3y & =9 \\
            \nonumber % Will skip the number in the next line
            x & =4 & y & =3
        \end{align}


    \section{Inline/Text Style Math}
        Notice that by sybstitution we get the equation \(f(x) = x^2 + 4x + 6\).
        The variable \(x\) is substituted by the value of \(x\) in the equation. \\[\baselineskip]
        Another way to do this si using the \verb|$|.Writting the equation inside the \verb|$|.
        In the form $f(x) = some_eqcuaiton$


    \section{Basic Notation}
        \underline{Arithmetic}
        \begin{align*}
            1 + 1 \\
            5 - 3 \\
            6 \cdot 4 \\
            6 \times 4 \\
            27 \div 9 \\
        \end{align*}

        \underline{Fractions}
        \begin{align*}
            % \frac{numerator}{denominator}
            \frac{x^2 + 3}{6 \times 4} \\
            % DISPLAY STYLE \dfrac{numerator}{denominator} \\ 
            \dfrac{23}{6 \div 12} \\ 
            % IN LINE STYLE \tfrac{numerator}{denominator} \\
            \tfrac{23}{6 \div 12} \\
        \end{align*}


    \section{Superscript and Subscript Notation}
        {\large \underline{Superscript}} \\
        Using a \textbf{caret} symbol will display a Superscript. \\
        Display Style \( \displaystyle e^{2y'} \) \\
        In line Style \( \textstyle e^{2x} \) \\[\baselineskip]

        {\large \underline{Subscript}} \\
        Using an \textbf{underscore} symbol will display a Subscript. \\
        Display Style \( \displaystyle x_2 \) \\
        In line Style \( \textstyle x_2 \) \\[\baselineskip]

        {\large \underline{Subscript and Superscript}} \\
        When we use \textbf{both at the same time}, the order doesn't matter. \\
        Example Code: \( \displaystyle y' = x^e_1  \) \\[\baselineskip] 

        Of course is possible to stack each other. \\
        Stacked Example: \( \displaystyle p_1^{x_0} \times x^e_2 \) \\
        Is preferable to use \textbf{Offset notation} because is much easier to read.
        Offset Example: \( \displaystyle {p_1}^{x_0} \times x^e_2 \) \\


    \section{Parentheses}
        


    \section{Some new section undiscovered yet}



\end{document}